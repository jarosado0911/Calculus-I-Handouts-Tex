\documentclass[10pt]{book}

%These tell TeX which packages to use.
\usepackage{array,epsfig}
\usepackage{amsmath}
\usepackage{amsfonts}
\usepackage{amssymb}
\usepackage{amsxtra}
\usepackage{amsthm}
\usepackage{mathrsfs}
\usepackage{color}
\usepackage{enumitem}

\usepackage{pgfplots}
\pgfplotsset{compat=1.6}

\pgfplotsset{soldot/.style={color=black,only marks,mark=*}} \pgfplotsset{holdot/.style={color=black,fill=white,only marks,mark=*}}

%Here I define some theorem styles and shortcut commands for symbols I use often
\theoremstyle{definition}
\newtheorem{defn}{Definition}
\newtheorem{thm}{Theorem}
\newtheorem{cor}{Corollary}
\newtheorem*{rmk}{Remark}
\newtheorem{lem}{Lemma}
\newtheorem*{joke}{Joke}
\newtheorem{ex}{Example}
\newtheorem*{soln}{Solution}
\newtheorem{prop}{Proposition}

\newcommand{\lra}{\longrightarrow}
\newcommand{\ra}{\rightarrow}
\newcommand{\surj}{\twoheadrightarrow}
\newcommand{\graph}{\mathrm{graph}}
\newcommand{\bb}[1]{\mathbb{#1}}
\newcommand{\Z}{\bb{Z}}
\newcommand{\Q}{\bb{Q}}
\newcommand{\R}{\bb{R}}
\newcommand{\C}{\bb{C}}
\newcommand{\N}{\bb{N}}
\newcommand{\M}{\mathbf{M}}
\newcommand{\m}{\mathbf{m}}
\newcommand{\MM}{\mathscr{M}}
\newcommand{\HH}{\mathscr{H}}
\newcommand{\Om}{\Omega}
\newcommand{\Ho}{\in\HH(\Om)}
\newcommand{\bd}{\partial}
\newcommand{\del}{\partial}
\newcommand{\bardel}{\overline\partial}
\newcommand{\textdf}[1]{\textbf{\textsf{#1}}\index{#1}}
\newcommand{\img}{\mathrm{img}}
\newcommand{\ip}[2]{\left\langle{#1},{#2}\right\rangle}
\newcommand{\inter}[1]{\mathrm{int}{#1}}
\newcommand{\exter}[1]{\mathrm{ext}{#1}}
\newcommand{\cl}[1]{\mathrm{cl}{#1}}
\newcommand{\ds}{\displaystyle}
\newcommand{\vol}{\mathrm{vol}}
\newcommand{\cnt}{\mathrm{ct}}
\newcommand{\osc}{\mathrm{osc}}
\newcommand{\LL}{\mathbf{L}}
\newcommand{\UU}{\mathbf{U}}
\newcommand{\support}{\mathrm{support}}
\newcommand{\AND}{\;\wedge\;}
\newcommand{\OR}{\;\vee\;}
\newcommand{\Oset}{\varnothing}
\newcommand{\st}{\ni}
\newcommand{\wh}{\widehat}
%Pagination stuff.
\setlength{\topmargin}{-0.75in}
\setlength{\oddsidemargin}{0in}
\setlength{\evensidemargin}{0in}
\setlength{\textheight}{9.in}
\setlength{\textwidth}{6.5in}
\pagestyle{empty}
\begin{document}
\begin{center}
{\Large Math 1041-012 \hspace{0.5cm} Classwork Finding Limits}
\end{center}
\vspace{0.2 cm}
\subsection*{Problem 1}
Evaluate each of the limits.
\begin{enumerate}[label=(\alph*)]
    \item $\displaystyle \lim_{x\rightarrow 3}\frac{3x^2+4x-1}{x}$\vspace{2.5cm}
    \item $\displaystyle \lim_{x\rightarrow 5}\frac{x^2-25}{5-x}$ \vspace{2.5cm}
    \item $\displaystyle \lim_{h\rightarrow 0}\frac{(4+h)^2-16}{h}$\vspace{2.5cm}
    \item $\displaystyle \lim_{x\rightarrow 9}\frac{\frac{1}{x}-\frac{1}{9}}{x-9}$\vspace{2.5cm}
    \item $\displaystyle \lim_{x\rightarrow 7}\frac{\sqrt{5x+1}-6}{x-7}$\vspace{2.5cm}
    \item $\displaystyle\lim_{x\rightarrow 3}(2x+|x-3|)$
\end{enumerate}
\pagebreak
\subsection*{Problem 2}
If
\[
f(x)=\begin{cases}
\sqrt{x-4} & x>4\\
8-2x & x<4
\end{cases}
\]
determine if $\lim_{x\rightarrow 4}$ exists.
\vspace{5cm}
\subsection*{Problem 3}
Evaluate the limit.
\[
\lim_{x\rightarrow 1}\frac{x^3-1}{x^2-1}
\]
\end{document}