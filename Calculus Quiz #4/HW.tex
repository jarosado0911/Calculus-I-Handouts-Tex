\documentclass[10pt]{book}

%These tell TeX which packages to use.
\usepackage{array,epsfig}
\usepackage{amsmath}
\usepackage{amsfonts}
\usepackage{amssymb}
\usepackage{amsxtra}
\usepackage{amsthm}
\usepackage{mathrsfs}
\usepackage{color}
\usepackage{enumitem}

\usepackage{pgfplots}
\pgfplotsset{compat=1.6}

\pgfplotsset{soldot/.style={color=black,only marks,mark=*}} \pgfplotsset{holdot/.style={color=black,fill=white,only marks,mark=*}}

%Here I define some theorem styles and shortcut commands for symbols I use often
\theoremstyle{definition}
\newtheorem{defn}{Definition}
\newtheorem{thm}{Theorem}
\newtheorem{cor}{Corollary}
\newtheorem*{rmk}{Remark}
\newtheorem{lem}{Lemma}
\newtheorem*{joke}{Joke}
\newtheorem{ex}{Example}
\newtheorem*{soln}{Solution}
\newtheorem{prop}{Proposition}

\newcommand{\lra}{\longrightarrow}
\newcommand{\ra}{\rightarrow}
\newcommand{\surj}{\twoheadrightarrow}
\newcommand{\graph}{\mathrm{graph}}
\newcommand{\bb}[1]{\mathbb{#1}}
\newcommand{\Z}{\bb{Z}}
\newcommand{\Q}{\bb{Q}}
\newcommand{\R}{\bb{R}}
\newcommand{\C}{\bb{C}}
\newcommand{\N}{\bb{N}}
\newcommand{\M}{\mathbf{M}}
\newcommand{\m}{\mathbf{m}}
\newcommand{\MM}{\mathscr{M}}
\newcommand{\HH}{\mathscr{H}}
\newcommand{\Om}{\Omega}
\newcommand{\Ho}{\in\HH(\Om)}
\newcommand{\bd}{\partial}
\newcommand{\del}{\partial}
\newcommand{\bardel}{\overline\partial}
\newcommand{\textdf}[1]{\textbf{\textsf{#1}}\index{#1}}
\newcommand{\img}{\mathrm{img}}
\newcommand{\ip}[2]{\left\langle{#1},{#2}\right\rangle}
\newcommand{\inter}[1]{\mathrm{int}{#1}}
\newcommand{\exter}[1]{\mathrm{ext}{#1}}
\newcommand{\cl}[1]{\mathrm{cl}{#1}}
\newcommand{\ds}{\displaystyle}
\newcommand{\vol}{\mathrm{vol}}
\newcommand{\cnt}{\mathrm{ct}}
\newcommand{\osc}{\mathrm{osc}}
\newcommand{\LL}{\mathbf{L}}
\newcommand{\UU}{\mathbf{U}}
\newcommand{\support}{\mathrm{support}}
\newcommand{\AND}{\;\wedge\;}
\newcommand{\OR}{\;\vee\;}
\newcommand{\Oset}{\varnothing}
\newcommand{\st}{\ni}
\newcommand{\wh}{\widehat}
%Pagination stuff.
\setlength{\topmargin}{-0.75in}
\setlength{\oddsidemargin}{0in}
\setlength{\evensidemargin}{0in}
\setlength{\textheight}{9.in}
\setlength{\textwidth}{6.5in}
\pagestyle{empty}
\begin{document}
\begin{flushleft}
Name:\underline{\hspace{13cm}}Date:\underline{\hspace{2cm}}
\end{flushleft}
\begin{center}
{\Large Math 1041-012 \hspace{0.5cm} Quiz \#4}
\end{center}
\vspace{0.2 cm}
\subsection*{Problem 1}Find each limit, if it exists. (Show work for partial credit)
\begin{enumerate}[label=(\alph*)]
    \item $\displaystyle\lim_{x\rightarrow-\infty}\frac{3}{x^4}$\vspace{3cm}
    \item $\displaystyle\lim_{x\rightarrow\infty}\frac{2-4x}{x+1}$\vspace{3cm}
    \item $\displaystyle\lim_{x\rightarrow\infty}\frac{x^3-4x+1}{2x-x^3}$\vspace{3cm}
    \item $\displaystyle\lim_{x\rightarrow-\infty}\frac{\sqrt{x^2+1}}{3+x}$\vspace{3cm}
    \item $\displaystyle\lim_{x\rightarrow-\infty}\cos (x^2)$
\end{enumerate}
\clearpage
\subsection*{Problem 2} Find the equation of the tangent line to the curve at the given point:
\[
y=\sqrt{x+1}\textrm{ at (3,2).}
\]
\vspace{6cm}
\subsection*{Problem 3} The limit below represents the derivative of some function $f(x)$ at some number $a$:
\[
\lim_{h\rightarrow 0}\frac{e^{-2+h}-e^{-2}}{h}\vspace{0.5cm}
\]
What are $a=\rule{1cm}{0.05mm}$ and $f(x)=\rule{1cm}{0.05mm} $.
\vspace{4cm}
\subsection*{Bonus}
Find the limit (show work for full credit, don't just write an answer!):
\[
\lim_{x\rightarrow 0^+}\tan^{-1}\left(\ln\left(x^{-1}\right)\right)
\]
\end{document}