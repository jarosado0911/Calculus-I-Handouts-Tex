\documentclass[10pt]{book}

%These tell TeX which packages to use.
\usepackage{array,epsfig}
\usepackage{amsmath}
\usepackage{amsfonts}
\usepackage{amssymb}
\usepackage{amsxtra}
\usepackage{amsthm}
\usepackage{mathrsfs}
\usepackage{color}
\usepackage{enumitem}

\usepackage{pgfplots}
\pgfplotsset{compat=1.6}

\pgfplotsset{soldot/.style={color=black,only marks,mark=*}} \pgfplotsset{holdot/.style={color=black,fill=white,only marks,mark=*}}

%Here I define some theorem styles and shortcut commands for symbols I use often
\theoremstyle{definition}
\newtheorem{defn}{Definition}
\newtheorem{thm}{Theorem}
\newtheorem{cor}{Corollary}
\newtheorem*{rmk}{Remark}
\newtheorem{lem}{Lemma}
\newtheorem*{joke}{Joke}
\newtheorem{ex}{Example}
\newtheorem*{soln}{Solution}
\newtheorem{prop}{Proposition}

\newcommand{\lra}{\longrightarrow}
\newcommand{\ra}{\rightarrow}
\newcommand{\surj}{\twoheadrightarrow}
\newcommand{\graph}{\mathrm{graph}}
\newcommand{\bb}[1]{\mathbb{#1}}
\newcommand{\Z}{\bb{Z}}
\newcommand{\Q}{\bb{Q}}
\newcommand{\R}{\bb{R}}
\newcommand{\C}{\bb{C}}
\newcommand{\N}{\bb{N}}
\newcommand{\M}{\mathbf{M}}
\newcommand{\m}{\mathbf{m}}
\newcommand{\MM}{\mathscr{M}}
\newcommand{\HH}{\mathscr{H}}
\newcommand{\Om}{\Omega}
\newcommand{\Ho}{\in\HH(\Om)}
\newcommand{\bd}{\partial}
\newcommand{\del}{\partial}
\newcommand{\bardel}{\overline\partial}
\newcommand{\textdf}[1]{\textbf{\textsf{#1}}\index{#1}}
\newcommand{\img}{\mathrm{img}}
\newcommand{\ip}[2]{\left\langle{#1},{#2}\right\rangle}
\newcommand{\inter}[1]{\mathrm{int}{#1}}
\newcommand{\exter}[1]{\mathrm{ext}{#1}}
\newcommand{\cl}[1]{\mathrm{cl}{#1}}
\newcommand{\ds}{\displaystyle}
\newcommand{\vol}{\mathrm{vol}}
\newcommand{\cnt}{\mathrm{ct}}
\newcommand{\osc}{\mathrm{osc}}
\newcommand{\LL}{\mathbf{L}}
\newcommand{\UU}{\mathbf{U}}
\newcommand{\support}{\mathrm{support}}
\newcommand{\AND}{\;\wedge\;}
\newcommand{\OR}{\;\vee\;}
\newcommand{\Oset}{\varnothing}
\newcommand{\st}{\ni}
\newcommand{\wh}{\widehat}
%Pagination stuff.
\setlength{\topmargin}{-0.75in}
\setlength{\oddsidemargin}{0in}
\setlength{\evensidemargin}{0in}
\setlength{\textheight}{9.in}
\setlength{\textwidth}{6.5in}
\pagestyle{empty}
\begin{document}
\begin{flushleft}
Name:\underline{\hspace{13cm}}Date:\underline{\hspace{2cm}}
\end{flushleft}
\begin{center}
{\Large Math 1041-012 \hspace{0.5cm} Classwork \#4}
\end{center}
\vspace{0.2 cm}
\subsection*{Problem 1}
Evaluate each limit. \textit{Hint: use Squeeze Theorem!}
\begin{enumerate}[label=(\alph*)]
    \item $\displaystyle\lim_{x\rightarrow 0+}\sqrt{x}e^{\cos(\pi/x)}$\vspace{2cm}
    \item 
    $\displaystyle \lim_{\theta\rightarrow \frac{\pi}{2}}\cos\theta \cos(\tan\theta))$\vspace{2cm}
    \item $\displaystyle\lim_{t\rightarrow 0}(2^t-1)\cos(1/t)$\vspace{2cm}
\end{enumerate}
\subsection*{Problem 2}
Find each limit if it exists!
\begin{enumerate}[label=(\alph*)]
    \item $\displaystyle\lim_{h\rightarrow 0^+}\frac{h^2-4}{h}$\vspace{1cm}
    \item $\displaystyle\lim_{x\rightarrow 2^-}\frac{\sqrt{3-x}}{x-2}$\vspace{1cm}
    \item$\displaystyle\lim_{y\rightarrow\pi}\frac{\cos y}{y-\pi}$\vspace{1cm}
    \item $\displaystyle\lim_{x\rightarrow 0}\frac{\sin x+x}{x}$\vspace{1cm}
    \item $\displaystyle\lim_{x\rightarrow 0}\frac{\sin x+2}{x}$
\end{enumerate}
\subsection*{Problem 3}
The graph of $f(x)$ is given below!
\begin{center}
\begin{tikzpicture}
\begin{axis}[
  axis x line=middle, axis y line=middle,
  ymin=-2, ymax=4, ytick={-2,...,4}, ylabel=$y$,
  xmin=-6, xmax=4, xtick={-6,...,4}, xlabel=$x$,
  axis line style={<->}, width=12cm,height=8cm
]
\addplot[<-,domain=-6:-4,black]{-0.5*x+cos(3*deg(x))};
\addplot[domain=-3:-1,black] {2*sin(deg(x))+3};
\addplot[domain=-4:-3,black]{x+5.7};
\addplot[domain=-1:1,black] {-0.5403*x^2+1.8574};
\addplot[domain=1:2,black]{2.31+1/-x};
\addplot[<->,domain=2:3.5,black]{3-1/(x-2)};
%\addplot[->,domain=6:10,black]{1/(x-6)};
\draw[dashed] (axis cs:2,4) -- (axis cs:2,-2);
\addplot[holdot] coordinates{(-4,2.8)(-2,1.18)(3,2)};
\addplot[soldot] coordinates{(2,1.81)(-4,1.7)(3,1)};
\end{axis}
\end{tikzpicture}
\end{center}
Where is $f(x)$ discontinuous? State the type of discontinuity and \textbf{\underline{why}}?
\end{document}