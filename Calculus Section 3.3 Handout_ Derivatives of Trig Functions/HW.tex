\documentclass[12pt]{book}

%These tell TeX which packages to use.
\usepackage{array,epsfig}
\usepackage{amsmath}
\usepackage{amsfonts}
\usepackage{amssymb}
\usepackage{amsxtra}
\usepackage{amsthm}
\usepackage{mathrsfs}
\usepackage{color}
\usepackage{enumitem}
%\usepackage{mdframed}
\usepackage[most]{tcolorbox}
\usepackage{pgfplots}
\pgfplotsset{compat=1.6}

\pgfplotsset{soldot/.style={color=black,only marks,mark=*}} \pgfplotsset{holdot/.style={color=black,fill=white,only marks,mark=*}}

%Here I define some theorem styles and shortcut commands for symbols I use often
\theoremstyle{definition}
\newtheorem{defn}{Definition}
\newtheorem{thm}{Theorem}
\newtheorem{cor}{Corollary}
\newtheorem*{rmk}{Remark}
\newtheorem{lem}{Lemma}
\newtheorem*{joke}{Joke}
\newtheorem{ex}{Example}
\newtheorem*{soln}{Solution}
\newtheorem{prop}{Proposition}

\newcommand{\lra}{\longrightarrow}
\newcommand{\ra}{\rightarrow}
\newcommand{\surj}{\twoheadrightarrow}
\newcommand{\graph}{\mathrm{graph}}
\newcommand{\bb}[1]{\mathbb{#1}}
\newcommand{\Z}{\bb{Z}}
\newcommand{\Q}{\bb{Q}}
\newcommand{\R}{\bb{R}}
\newcommand{\C}{\bb{C}}
\newcommand{\N}{\bb{N}}
\newcommand{\M}{\mathbf{M}}
\newcommand{\m}{\mathbf{m}}
\newcommand{\MM}{\mathscr{M}}
\newcommand{\HH}{\mathscr{H}}
\newcommand{\Om}{\Omega}
\newcommand{\Ho}{\in\HH(\Om)}
\newcommand{\bd}{\partial}
\newcommand{\del}{\partial}
\newcommand{\bardel}{\overline\partial}
\newcommand{\textdf}[1]{\textbf{\textsf{#1}}\index{#1}}
\newcommand{\img}{\mathrm{img}}
\newcommand{\ip}[2]{\left\langle{#1},{#2}\right\rangle}
\newcommand{\inter}[1]{\mathrm{int}{#1}}
\newcommand{\exter}[1]{\mathrm{ext}{#1}}
\newcommand{\cl}[1]{\mathrm{cl}{#1}}
\newcommand{\ds}{\displaystyle}
\newcommand{\vol}{\mathrm{vol}}
\newcommand{\cnt}{\mathrm{ct}}
\newcommand{\osc}{\mathrm{osc}}
\newcommand{\LL}{\mathbf{L}}
\newcommand{\UU}{\mathbf{U}}
\newcommand{\support}{\mathrm{support}}
\newcommand{\AND}{\;\wedge\;}
\newcommand{\OR}{\;\vee\;}
\newcommand{\Oset}{\varnothing}
\newcommand{\st}{\ni}
\newcommand{\wh}{\widehat}
%Pagination stuff.
\setlength{\topmargin}{-0.75in}
\setlength{\oddsidemargin}{0in}
\setlength{\evensidemargin}{0in}
\setlength{\textheight}{9.in}
\setlength{\textwidth}{6.5in}
\pagestyle{empty}
\begin{document}
\begin{flushleft}
Name:\underline{\hspace{13cm}}Date:\underline{\hspace{2cm}}
\end{flushleft}
\begin{center}
{\Large Math 1041-012 \hspace{0.5cm} Section 3.3: Trigonometric Derivatives}
\end{center}
%\vspace{0.2 cm}
\begin{tcolorbox}
\subsection*{Two Important Limits!}
\begin{itemize}
    \item $\displaystyle\lim_{x\rightarrow 0}\frac{\sin x}{x}=1$
    \item $\displaystyle \lim_{x\rightarrow 0}\frac{\cos x - 1}{x}=0$
\end{itemize}
These limits help us find the derivatives of sine and cosine!\\
Note: Recall $\sin(a+b)=\sin a \cos b+\cos a\sin b$.
\end{tcolorbox}
\subsection*{Find the derivative of Sine?}
To find $(\sin x)'$ we go back to the definition of derivative:
\begin{align*}
    \frac{d}{dx}\sin x = \lim_{h\rightarrow 0}\frac{\sin(x+h)-\sin x}{h}=
\end{align*}
\vspace{6cm}
\begin{tcolorbox}
\subsection*{Fundamental Trig Rules}
\begin{itemize}
    \item $\displaystyle\frac{d}{dx}\sin x=\cos x$
    \item $\displaystyle\frac{d}{dx}\cos x=-\sin x$
\end{itemize}
We use these fundamental rules to compute the derivatives of other trig functions!
\end{tcolorbox}
\raggedbottom
\clearpage
\subsection*{Example: Find the derivative of tangent!}
Find $\frac{d}{dx}\tan x$\vspace{4cm}
\begin{tcolorbox}
\subsection*{Other Trig Rules (memorize me!!!)}
\begin{itemize}
    \item $\displaystyle\frac{d}{dx}\cot x=$
    \item $\displaystyle\frac{d}{dx}\sec x=$
    \item $\displaystyle\frac{d}{dx}\csc x=$
\end{itemize}
\end{tcolorbox}
\subsection*{Combining all derivative rules!}
\begin{itemize}
    \item[(a)] Find $(x^2\sin x)'$\vspace{4cm}
    \item[(b)] Find $\frac{d}{dx}\frac{\sec x}{1+\tan x}$.\vspace{4cm}
    \item[(c)] The position of an object at the end of a vertical spring is given by $s=f(t)=4\cos t$. Find the velocity and acceleration at time $t$.
\end{itemize}
\raggedbottom
\clearpage
\subsection*{A little out of place, but let us revisit limits!}
\begin{itemize}
    \item[(a)] Find $\displaystyle \lim_{x\rightarrow 0}\frac{\sin 7x}{4x}$\vspace{5cm}
    \item[(b)] Find $\displaystyle \lim_{x\rightarrow 0}x\cot x$\vspace{5cm}
    \item[(c)]Find $\displaystyle \lim_{\theta\rightarrow 0}\frac{\cos \theta -1}{\sin \theta}$
\end{itemize}
\end{document}