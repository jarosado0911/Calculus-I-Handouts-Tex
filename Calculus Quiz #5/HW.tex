\documentclass[10pt]{book}

%These tell TeX which packages to use.
\usepackage{array,epsfig}
\usepackage{amsmath}
\usepackage{amsfonts}
\usepackage{amssymb}
\usepackage{amsxtra}
\usepackage{amsthm}
\usepackage{mathrsfs}
\usepackage{color}
\usepackage{enumitem}

\usepackage{pgfplots}
\pgfplotsset{compat=1.6}

\pgfplotsset{soldot/.style={color=black,only marks,mark=*}} \pgfplotsset{holdot/.style={color=black,fill=white,only marks,mark=*}}

%Here I define some theorem styles and shortcut commands for symbols I use often
\theoremstyle{definition}
\newtheorem{defn}{Definition}
\newtheorem{thm}{Theorem}
\newtheorem{cor}{Corollary}
\newtheorem*{rmk}{Remark}
\newtheorem{lem}{Lemma}
\newtheorem*{joke}{Joke}
\newtheorem{ex}{Example}
\newtheorem*{soln}{Solution}
\newtheorem{prop}{Proposition}

\newcommand{\lra}{\longrightarrow}
\newcommand{\ra}{\rightarrow}
\newcommand{\surj}{\twoheadrightarrow}
\newcommand{\graph}{\mathrm{graph}}
\newcommand{\bb}[1]{\mathbb{#1}}
\newcommand{\Z}{\bb{Z}}
\newcommand{\Q}{\bb{Q}}
\newcommand{\R}{\bb{R}}
\newcommand{\C}{\bb{C}}
\newcommand{\N}{\bb{N}}
\newcommand{\M}{\mathbf{M}}
\newcommand{\m}{\mathbf{m}}
\newcommand{\MM}{\mathscr{M}}
\newcommand{\HH}{\mathscr{H}}
\newcommand{\Om}{\Omega}
\newcommand{\Ho}{\in\HH(\Om)}
\newcommand{\bd}{\partial}
\newcommand{\del}{\partial}
\newcommand{\bardel}{\overline\partial}
\newcommand{\textdf}[1]{\textbf{\textsf{#1}}\index{#1}}
\newcommand{\img}{\mathrm{img}}
\newcommand{\ip}[2]{\left\langle{#1},{#2}\right\rangle}
\newcommand{\inter}[1]{\mathrm{int}{#1}}
\newcommand{\exter}[1]{\mathrm{ext}{#1}}
\newcommand{\cl}[1]{\mathrm{cl}{#1}}
\newcommand{\ds}{\displaystyle}
\newcommand{\vol}{\mathrm{vol}}
\newcommand{\cnt}{\mathrm{ct}}
\newcommand{\osc}{\mathrm{osc}}
\newcommand{\LL}{\mathbf{L}}
\newcommand{\UU}{\mathbf{U}}
\newcommand{\support}{\mathrm{support}}
\newcommand{\AND}{\;\wedge\;}
\newcommand{\OR}{\;\vee\;}
\newcommand{\Oset}{\varnothing}
\newcommand{\st}{\ni}
\newcommand{\wh}{\widehat}
%Pagination stuff.
\setlength{\topmargin}{-0.75in}
\setlength{\oddsidemargin}{0in}
\setlength{\evensidemargin}{0in}
\setlength{\textheight}{9.in}
\setlength{\textwidth}{6.5in}
\pagestyle{empty}
\begin{document}
\begin{flushleft}
Name:\underline{\hspace{13cm}}Date:\underline{\hspace{2cm}}
\end{flushleft}
\begin{center}
{\Large Math 1041-012 \hspace{0.5cm} Quiz \#5}
\end{center}
\vspace{0.2 cm}
\subsection*{Problem 1} The graph of $f(x)$ is given below!
\begin{center}
\begin{tikzpicture}
\begin{axis}[
  axis x line=middle, axis y line=middle,
  ymin=-2, ymax=4, ytick={-2,...,4}, ylabel=$y$,
  xmin=-6, xmax=4, xtick={-6,...,4}, xlabel=$x$,
  axis line style={<->}, width=12cm,height=8cm
]
\addplot[<-,domain=-6:-4,black]{-0.5*x+cos(3*deg(x))};
\addplot[domain=-3:-1,black] {2*sin(deg(x))+3};
\addplot[domain=-4:-3,black]{x+5.7};
\addplot[domain=-1:1,black] {-0.5403*x^2+1.8574};
\addplot[domain=1:2,black]{2.31+1/-x};
\addplot[<->,domain=2:3.5,black]{3-1/(x-2)};
%\addplot[->,domain=6:10,black]{1/(x-6)};
\draw[dashed] (axis cs:2,4) -- (axis cs:2,-2);
\addplot[holdot] coordinates{(-4,2.8)};
\addplot[soldot] coordinates{(2,1.81)(-4,1.7)};
\end{axis}
\end{tikzpicture}
\end{center}
\begin{enumerate}[label=(\alph*)]
    \item State the intervals on which $f(x)$ is continuous!\\[4pt]
    \item State the intervals on which $f(x)$ is differentiable.\\[4pt]
    \item At what $x$ values is $f(x)$ non-differentiable?
\end{enumerate}
\subsection*{Problem 2} Below is the limit definition for the derivative of a function!
\[
\boxed{f'(x)=\lim_{h\rightarrow 0}\frac{f(x+h)-f(x)}{h}}
\]
Use the definition above to find the derivative of $H(x)=\sqrt{3-x}$.
\clearpage
\subsection*{Problem 3} Use derivative rules to differentiate each function.
\begin{enumerate}[label=(\alph*)]
    \item $\displaystyle f(x)=2x^{11}-5x^2+\sqrt{x^3}$\vspace{3cm}
    \item $\displaystyle\frac{d}{dw}(w+4w(w^2-1))$\vspace{3cm}
    \item $\displaystyle \frac{d}{dx}(2^\pi)$\vspace{3cm}
    \item $\displaystyle g(x)=\frac{3x^{1/2}+5x^{3/2}+x+3}{x\sqrt{x}}$\vspace{4cm}
\end{enumerate}
\subsection*{Bonus} Find the third derivative below!
\[
\frac{d^3}{d\pi^3} (\pi^e)
\]
\end{document}
