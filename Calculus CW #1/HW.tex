\documentclass[12pt]{book}

%These tell TeX which packages to use.
\usepackage{array,epsfig}
\usepackage{amsmath}
\usepackage{amsfonts}
\usepackage{amssymb}
\usepackage{amsxtra}
\usepackage{amsthm}
\usepackage{mathrsfs}
\usepackage{color}
\usepackage{enumitem}
%Here I define some theorem styles and shortcut commands for symbols I use often
\theoremstyle{definition}
\newtheorem{defn}{Definition}
\newtheorem{thm}{Theorem}
\newtheorem{cor}{Corollary}
\newtheorem*{rmk}{Remark}
\newtheorem{lem}{Lemma}
\newtheorem*{joke}{Joke}
\newtheorem{ex}{Example}
\newtheorem*{soln}{Solution}
\newtheorem{prop}{Proposition}

\newcommand{\lra}{\longrightarrow}
\newcommand{\ra}{\rightarrow}
\newcommand{\surj}{\twoheadrightarrow}
\newcommand{\graph}{\mathrm{graph}}
\newcommand{\bb}[1]{\mathbb{#1}}
\newcommand{\Z}{\bb{Z}}
\newcommand{\Q}{\bb{Q}}
\newcommand{\R}{\bb{R}}
\newcommand{\C}{\bb{C}}
\newcommand{\N}{\bb{N}}
\newcommand{\M}{\mathbf{M}}
\newcommand{\m}{\mathbf{m}}
\newcommand{\MM}{\mathscr{M}}
\newcommand{\HH}{\mathscr{H}}
\newcommand{\Om}{\Omega}
\newcommand{\Ho}{\in\HH(\Om)}
\newcommand{\bd}{\partial}
\newcommand{\del}{\partial}
\newcommand{\bardel}{\overline\partial}
\newcommand{\textdf}[1]{\textbf{\textsf{#1}}\index{#1}}
\newcommand{\img}{\mathrm{img}}
\newcommand{\ip}[2]{\left\langle{#1},{#2}\right\rangle}
\newcommand{\inter}[1]{\mathrm{int}{#1}}
\newcommand{\exter}[1]{\mathrm{ext}{#1}}
\newcommand{\cl}[1]{\mathrm{cl}{#1}}
\newcommand{\ds}{\displaystyle}
\newcommand{\vol}{\mathrm{vol}}
\newcommand{\cnt}{\mathrm{ct}}
\newcommand{\osc}{\mathrm{osc}}
\newcommand{\LL}{\mathbf{L}}
\newcommand{\UU}{\mathbf{U}}
\newcommand{\support}{\mathrm{support}}
\newcommand{\AND}{\;\wedge\;}
\newcommand{\OR}{\;\vee\;}
\newcommand{\Oset}{\varnothing}
\newcommand{\st}{\ni}
\newcommand{\wh}{\widehat}

%Pagination stuff.
\setlength{\topmargin}{-0.75in}
\setlength{\oddsidemargin}{0in}
\setlength{\evensidemargin}{0in}
\setlength{\textheight}{9.in}
\setlength{\textwidth}{6.5in}
\pagestyle{empty}
\begin{document}
\begin{center}
{\Large Math 1041-012 \hspace{0.5cm} Classwork \#1}
\end{center}
\vspace{0.2 cm}
\subsection*{Solve each of the following for $x$.}
\begin{enumerate}
\item $11x=2x^2+12$\\[2pt]
\item $4x^2=8x$\\[2pt]
\item $\frac{x}{x-2}-3=0$\\[2pt]
\end{enumerate}
\subsection*{Find $\frac{f(x+h)-f(x)}{h}$, $h\neq 0$ for the following functions. Simplify Fully!}
\begin{enumerate}[resume]
    \item $f(x)=\frac{1}{x+3}$\\[16pt]
    \item $f(x)= -3x^2+4x+6$\\[16pt]
\end{enumerate}
\subsection*{Simplify the following expressions}
\begin{enumerate}[resume]
    \item $(e^{5x})^2$
    \item $7^{4\log_7 3}$
    \item $e^{\ln 2}-3\ln \sqrt[3]{e}$
    \item $\log_9 4+\log_9\left(\frac{81}{4}\right)$
\end{enumerate}
\subsection*{Solve the following equations}
\begin{enumerate}[resume]
    \item $\ln x-\ln(x+3)=-1$\\[4pt]
    \item $e^{7-4x}=6$\\[4pt]
    \item $\ln (x^2-1)=3$\\[4pt]
    \item $e^{2x}-3e^x+2=0$\\[4pt]
\end{enumerate}
\subsection*{Find the exact values of each of the following expressions.}
\begin{enumerate}[resume]
    \item $\sin\frac{7\pi}{3}$\\[4pt]
    \item $2\sin\frac{\pi}{6}+4\cos\frac{\pi}{3}$\\[4pt]
    \item $\tan\frac{2\pi}{3}\cot\frac{\pi}{3}$\\[4pt]
    \item $\csc\frac{3\pi}{4}$\\[4pt]
\end{enumerate}
\subsection*{Find all solutions to the following equations in the interval $[0,2\pi)$.}
\begin{enumerate}[resume]
    \item $2\cos x-1=0$\\[8pt]
    \item $2\sin^2x=1$\\[8pt]
    \item $\sin 2x=\cos x$\\[8pt]
    \item $2+\cos 2x=3\cos x$
\end{enumerate}
\end{document}