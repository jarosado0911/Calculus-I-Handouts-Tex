\documentclass[10pt]{book}

%These tell TeX which packages to use.
\usepackage{array,epsfig}
\usepackage{amsmath}
\usepackage{amsfonts}
\usepackage{amssymb}
\usepackage{amsxtra}
\usepackage{amsthm}
\usepackage{mathrsfs}
\usepackage{color}
\usepackage{enumitem}
%\usepackage{mdframed}
\usepackage[most]{tcolorbox}
\usepackage{pgfplots}
\pgfplotsset{compat=1.6}

\pgfplotsset{soldot/.style={color=black,only marks,mark=*}} \pgfplotsset{holdot/.style={color=black,fill=white,only marks,mark=*}}

%Here I define some theorem styles and shortcut commands for symbols I use often
\theoremstyle{definition}
\newtheorem{defn}{Definition}
\newtheorem{thm}{Theorem}
\newtheorem{cor}{Corollary}
\newtheorem*{rmk}{Remark}
\newtheorem{lem}{Lemma}
\newtheorem*{joke}{Joke}
\newtheorem{ex}{Example}
\newtheorem*{soln}{Solution}
\newtheorem{prop}{Proposition}

\newcommand{\lra}{\longrightarrow}
\newcommand{\ra}{\rightarrow}
\newcommand{\surj}{\twoheadrightarrow}
\newcommand{\graph}{\mathrm{graph}}
\newcommand{\bb}[1]{\mathbb{#1}}
\newcommand{\Z}{\bb{Z}}
\newcommand{\Q}{\bb{Q}}
\newcommand{\R}{\bb{R}}
\newcommand{\C}{\bb{C}}
\newcommand{\N}{\bb{N}}
\newcommand{\M}{\mathbf{M}}
\newcommand{\m}{\mathbf{m}}
\newcommand{\MM}{\mathscr{M}}
\newcommand{\HH}{\mathscr{H}}
\newcommand{\Om}{\Omega}
\newcommand{\Ho}{\in\HH(\Om)}
\newcommand{\bd}{\partial}
\newcommand{\del}{\partial}
\newcommand{\bardel}{\overline\partial}
\newcommand{\textdf}[1]{\textbf{\textsf{#1}}\index{#1}}
\newcommand{\img}{\mathrm{img}}
\newcommand{\ip}[2]{\left\langle{#1},{#2}\right\rangle}
\newcommand{\inter}[1]{\mathrm{int}{#1}}
\newcommand{\exter}[1]{\mathrm{ext}{#1}}
\newcommand{\cl}[1]{\mathrm{cl}{#1}}
\newcommand{\ds}{\displaystyle}
\newcommand{\vol}{\mathrm{vol}}
\newcommand{\cnt}{\mathrm{ct}}
\newcommand{\osc}{\mathrm{osc}}
\newcommand{\LL}{\mathbf{L}}
\newcommand{\UU}{\mathbf{U}}
\newcommand{\support}{\mathrm{support}}
\newcommand{\AND}{\;\wedge\;}
\newcommand{\OR}{\;\vee\;}
\newcommand{\Oset}{\varnothing}
\newcommand{\st}{\ni}
\newcommand{\wh}{\widehat}
%Pagination stuff.
\setlength{\topmargin}{-0.75in}
\setlength{\oddsidemargin}{0in}
\setlength{\evensidemargin}{0in}
\setlength{\textheight}{9.in}
\setlength{\textwidth}{6.5in}
\pagestyle{empty}
\begin{document}
\begin{flushleft}
Name:\underline{\hspace{13cm}}Date:\underline{\hspace{2cm}}
\end{flushleft}
\begin{center}
{\Large Math 1041-012 \hspace{0.5cm} Section 3.9: Related Rates}
\end{center}
%\vspace{0.2 cm}

\begin{tcolorbox}
\subsection*{Problem Solving for Related Rates}
\begin{enumerate}
    \item Read and underline key words in the word problem. What are they looking for? $dy/dt$ or $dV/dt$ or $dS/dt$...
    \item Draw a figure if necessary.
    \item There is usually a formula that goes with the problem.
    \item Relate given numbers to the formula, you may need to reduce the number of variable.
    \item Differentiate the changing quantities with respect to time.
    \item Solve for ``$dy/dt$".
\end{enumerate}
\end{tcolorbox}
\subsection*{Example 1: The balloon problem again}
Air is being pumped into a spherical balloon so that its volume increases at a rate of 100 centimeters per second. How fast is the radius of the balloon increasing when the diameter is 50 cm?
\begin{enumerate}[label=(\alph*)]
    \item Make a sketch of the problem. Indicate dimensions with variables. What formula goes with the picture?\vspace{4cm} 
    \item In the problem what are the quantities that are changing? What are you asked to find?\vspace{2cm}
    \item What quantities do they give you?\vspace{2cm}
    \item Differentiate your formula with respect to time. Then use given quantities to solve the problem.
\end{enumerate}
\clearpage
\subsection*{Example 2: The ladder problem}
A 10 foot ladder leans against a vertical wall. If the bottom of the ladder slides away from the wall at a rate of 1 foot per second, how fast is the top of the ladder sliding down the wall when the bottom of the ladder is six feet from the wall?
\begin{enumerate}[label=(\alph*)]
    \item Make a sketch of the problem. Indicate dimensions with variables. What formula goes with the picture?\vspace{4cm} 
    \item In the problem what are the quantities that are changing? What are you asked to find?\vspace{2cm}
    \item What quantities do they give you?\vspace{2cm}
    \item Differentiate your formula with respect to time. Then use given quantities to solve the problem.
\end{enumerate}
\clearpage
\subsection*{Example 3: The conical water tank} A water tank has the shape of an inverted circular cone with base radius 2 meters and height 4 meters. If the water is being pumped into the tank at a rate of 3 cubic meters per min, find the rate at which the water level is rising when the water is 3 meters deep.
\clearpage
\subsection*{Example 4: Traveling Cars} Car A is traveling west at 50 miles per hour and Car B is traveling north at 60 miles per hour. Both cars are heading to the same intersection of the two roads. At what rate are the cars approaching each other when Car A is 0.3 miles from the intersection and Car B is 0.4 miles from the intersection?
\end{document}