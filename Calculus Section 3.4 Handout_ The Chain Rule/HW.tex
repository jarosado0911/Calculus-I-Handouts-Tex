\documentclass[12pt]{book}

%These tell TeX which packages to use.
\usepackage{array,epsfig}
\usepackage{amsmath}
\usepackage{amsfonts}
\usepackage{amssymb}
\usepackage{amsxtra}
\usepackage{amsthm}
\usepackage{mathrsfs}
\usepackage{color}
\usepackage{enumitem}
%\usepackage{mdframed}
\usepackage[most]{tcolorbox}
\usepackage{pgfplots}
\pgfplotsset{compat=1.6}

\pgfplotsset{soldot/.style={color=black,only marks,mark=*}} \pgfplotsset{holdot/.style={color=black,fill=white,only marks,mark=*}}

%Here I define some theorem styles and shortcut commands for symbols I use often
\theoremstyle{definition}
\newtheorem{defn}{Definition}
\newtheorem{thm}{Theorem}
\newtheorem{cor}{Corollary}
\newtheorem*{rmk}{Remark}
\newtheorem{lem}{Lemma}
\newtheorem*{joke}{Joke}
\newtheorem{ex}{Example}
\newtheorem*{soln}{Solution}
\newtheorem{prop}{Proposition}

\newcommand{\lra}{\longrightarrow}
\newcommand{\ra}{\rightarrow}
\newcommand{\surj}{\twoheadrightarrow}
\newcommand{\graph}{\mathrm{graph}}
\newcommand{\bb}[1]{\mathbb{#1}}
\newcommand{\Z}{\bb{Z}}
\newcommand{\Q}{\bb{Q}}
\newcommand{\R}{\bb{R}}
\newcommand{\C}{\bb{C}}
\newcommand{\N}{\bb{N}}
\newcommand{\M}{\mathbf{M}}
\newcommand{\m}{\mathbf{m}}
\newcommand{\MM}{\mathscr{M}}
\newcommand{\HH}{\mathscr{H}}
\newcommand{\Om}{\Omega}
\newcommand{\Ho}{\in\HH(\Om)}
\newcommand{\bd}{\partial}
\newcommand{\del}{\partial}
\newcommand{\bardel}{\overline\partial}
\newcommand{\textdf}[1]{\textbf{\textsf{#1}}\index{#1}}
\newcommand{\img}{\mathrm{img}}
\newcommand{\ip}[2]{\left\langle{#1},{#2}\right\rangle}
\newcommand{\inter}[1]{\mathrm{int}{#1}}
\newcommand{\exter}[1]{\mathrm{ext}{#1}}
\newcommand{\cl}[1]{\mathrm{cl}{#1}}
\newcommand{\ds}{\displaystyle}
\newcommand{\vol}{\mathrm{vol}}
\newcommand{\cnt}{\mathrm{ct}}
\newcommand{\osc}{\mathrm{osc}}
\newcommand{\LL}{\mathbf{L}}
\newcommand{\UU}{\mathbf{U}}
\newcommand{\support}{\mathrm{support}}
\newcommand{\AND}{\;\wedge\;}
\newcommand{\OR}{\;\vee\;}
\newcommand{\Oset}{\varnothing}
\newcommand{\st}{\ni}
\newcommand{\wh}{\widehat}
%Pagination stuff.
\setlength{\topmargin}{-0.75in}
\setlength{\oddsidemargin}{0in}
\setlength{\evensidemargin}{0in}
\setlength{\textheight}{9.in}
\setlength{\textwidth}{6.5in}
\pagestyle{empty}
\begin{document}
\begin{flushleft}
Name:\underline{\hspace{13cm}}Date:\underline{\hspace{2cm}}
\end{flushleft}
\begin{center}
{\Large Math 1041-012 \hspace{0.5cm} Section 3.4: The Chain Rule}
\end{center}
%\vspace{0.2 cm}
\subsection*{Motivating Example}
How do we find
\[
\frac{d}{dx}\sqrt{x^2+1}=??
\]
\vspace{2cm}
\begin{tcolorbox}
\subsection*{The Chain Rule}
Think of it as the ``Outside-Inside Rule''.\\ \\
If $g$ is differentiable at $x$ and $f$ is differentiable at $g(x)$, then the composite function $F=f\circ g$ which is $F(x)=f(g(x))$ is differentiable at $x$ and is given by
\[
F'(x)=f'(g(x))\cdot g'(x)
\]
\vspace{2cm}
\end{tcolorbox}
\subsection*{Many Examples!}
For each problem first identify the ``outside'' function and the ``inside'' function. Then use the Chain Rule to find the derivative!
\begin{itemize}
    \item[(a)] $F(x)=\sqrt{\sin x}$\vspace{3cm}
    \item[(b)] $(\sin(x^2))'$\vspace{3cm}
    \item[(c)] $\frac{d}{dx}\sin^2(x)$\vspace{3cm}
    \item[(d)] $\frac{d}{d\theta}\sin(\cos\theta)$\vspace{3cm}
\end{itemize}
\begin{tcolorbox}
\subsection*{Power-Chain Rule} If $y=[f(x)]^n$ then
\[
y'=n[f(x)]^{n-1}\cdot f'(x)
\]
\end{tcolorbox}
\subsection*{Even More Examples!!}
Use the power chain rule to differentiate each function!
\begin{itemize}
    \item[(a)]$y=(x^3-1)^{100}$\vspace{4cm}
    \item[(b)]$f(x)=\frac{1}{\sqrt[3]{x^2+x+1}}$\vspace{3cm}\raggedbottom\clearpage
    \item[(c)]$g(t)=\left(\frac{t-2}{2t+1}\right)^9$\vspace{4cm}
    \item[(d)]$y=(2x+1)^5(x^3-x+1)^6$\vspace{3cm}
\end{itemize}
\subsection*{Miscellaneous Examples}
Find the derivative of each!
\begin{itemize}
    \item[(a)] $y=e^{\sin x}$\vspace{3cm}
    \item[(b)] $f(x)=\sin(\cos(\tan x)))$\vspace{5cm}
    \item[(c)] $y=e^{\sin(3\theta)}$\vspace{3cm}
\end{itemize}
\raggedbottom
\clearpage
\begin{tcolorbox}
\subsection*{A random derivative rule! Okay}
\[
\frac{d}{dx}b^x=b^x\ln b
\]
(what happens if you let $b=e$?)
\end{tcolorbox}
\subsection*{A few more examples}
\begin{itemize}
    \item[(a)] $y=2^{\tan x}$\vspace{5cm}
    \item[(b)] $f(x)=\pi^{2^x}$
\end{itemize}
\end{document}