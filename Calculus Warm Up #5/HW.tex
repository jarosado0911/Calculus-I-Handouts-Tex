\documentclass[10pt]{book}

%These tell TeX which packages to use.
\usepackage{array,epsfig}
\usepackage{amsmath}
\usepackage{amsfonts}
\usepackage{amssymb}
\usepackage{amsxtra}
\usepackage{amsthm}
\usepackage{mathrsfs}
\usepackage{color}
\usepackage{enumitem}
\usepackage{tikz}
\usetikzlibrary{arrows}
\usepackage{pgfplots}
\pgfplotsset{compat=1.6}

\pgfplotsset{soldot/.style={color=black,only marks,mark=*}} \pgfplotsset{holdot/.style={color=black,fill=white,only marks,mark=*}}

%Here I define some theorem styles and shortcut commands for symbols I use often
\theoremstyle{definition}
\newtheorem{defn}{Definition}
\newtheorem{thm}{Theorem}
\newtheorem{cor}{Corollary}
\newtheorem*{rmk}{Remark}
\newtheorem{lem}{Lemma}
\newtheorem*{joke}{Joke}
\newtheorem{ex}{Example}
\newtheorem*{soln}{Solution}
\newtheorem{prop}{Proposition}

\newcommand{\lra}{\longrightarrow}
\newcommand{\ra}{\rightarrow}
\newcommand{\surj}{\twoheadrightarrow}
\newcommand{\graph}{\mathrm{graph}}
\newcommand{\bb}[1]{\mathbb{#1}}
\newcommand{\Z}{\bb{Z}}
\newcommand{\Q}{\bb{Q}}
\newcommand{\R}{\bb{R}}
\newcommand{\C}{\bb{C}}
\newcommand{\N}{\bb{N}}
\newcommand{\M}{\mathbf{M}}
\newcommand{\m}{\mathbf{m}}
\newcommand{\MM}{\mathscr{M}}
\newcommand{\HH}{\mathscr{H}}
\newcommand{\Om}{\Omega}
\newcommand{\Ho}{\in\HH(\Om)}
\newcommand{\bd}{\partial}
\newcommand{\del}{\partial}
\newcommand{\bardel}{\overline\partial}
\newcommand{\textdf}[1]{\textbf{\textsf{#1}}\index{#1}}
\newcommand{\img}{\mathrm{img}}
\newcommand{\ip}[2]{\left\langle{#1},{#2}\right\rangle}
\newcommand{\inter}[1]{\mathrm{int}{#1}}
\newcommand{\exter}[1]{\mathrm{ext}{#1}}
\newcommand{\cl}[1]{\mathrm{cl}{#1}}
\newcommand{\ds}{\displaystyle}
\newcommand{\vol}{\mathrm{vol}}
\newcommand{\cnt}{\mathrm{ct}}
\newcommand{\osc}{\mathrm{osc}}
\newcommand{\LL}{\mathbf{L}}
\newcommand{\UU}{\mathbf{U}}
\newcommand{\support}{\mathrm{support}}
\newcommand{\AND}{\;\wedge\;}
\newcommand{\OR}{\;\vee\;}
\newcommand{\Oset}{\varnothing}
\newcommand{\st}{\ni}
\newcommand{\wh}{\widehat}
%Pagination stuff.
\setlength{\topmargin}{-0.75in}
\setlength{\oddsidemargin}{0in}
\setlength{\evensidemargin}{0in}
\setlength{\textheight}{9.in}
\setlength{\textwidth}{6.5in}
\pagestyle{empty}
\begin{document}
%\begin{flushleft}
%Name:\underline{\hspace{13cm}}Date:\underline{\hspace{2cm}}
%\end{flushleft}
\begin{center}
{\Large Math 1041-012 \hspace{0.5cm} Warm-Up}
\end{center}
\subsection*{Problem 1} Find the limits!
\begin{enumerate}[label=(\alph*)]
    \item $\displaystyle\lim_{x\rightarrow-\infty}(x-2)^4(x+1)^3(x^2+1)$\vspace{2cm}
    \item $\displaystyle \lim_{x\rightarrow 1^+}\frac{\sqrt{x^2+3}}{x-1}$\vspace{2cm}
    \item $\displaystyle \lim_{x\rightarrow -\infty}\frac{\sqrt{x^2+3}}{x-1}$\vspace{2cm}
\end{enumerate}
\vspace{1cm}
\begin{center}
{\Large Math 1041-012 \hspace{0.5cm} Warm-Up}
\end{center}
\subsection*{Problem 1} Find the limits!
\begin{enumerate}[label=(\alph*)]
    \item $\displaystyle\lim_{x\rightarrow-\infty}(x-2)^4(x+1)^3(x^2+1)$\vspace{2cm}
    \item $\displaystyle \lim_{x\rightarrow 1^+}\frac{\sqrt{x^2+3}}{x-1}$\vspace{2cm}
    \item $\displaystyle \lim_{x\rightarrow -\infty}\frac{\sqrt{x^2+3}}{x-1}$\vspace{2cm}
\end{enumerate}
\clearpage
\subsection*{Problem 2} The limit below is the derivative of $f(x)$ at $a$.
\[
\lim_{h\rightarrow 0}\frac{(2+h)^3-8}{h}.
\]
What is $a=\ \ \ \ \ $ and $f(x)=\ \ \ \ \ $
\vspace{10cm}
\subsection*{Problem 2} The limit below is the derivative of $f(x)$ at $a$.
\[
\lim_{h\rightarrow 0}\frac{(2+h)^3-8}{h}.
\]
What is $a=\ \ \ \ \ $ and $f(x)=\ \ \ \ \ $
\end{document}