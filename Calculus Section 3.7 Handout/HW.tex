\documentclass[10pt]{book}

%These tell TeX which packages to use.
\usepackage{array,epsfig}
\usepackage{amsmath}
\usepackage{amsfonts}
\usepackage{amssymb}
\usepackage{amsxtra}
\usepackage{amsthm}
\usepackage{mathrsfs}
\usepackage{color}
\usepackage{enumitem}
%\usepackage{mdframed}
\usepackage[most]{tcolorbox}
\usepackage{pgfplots}
\pgfplotsset{compat=1.6}

\pgfplotsset{soldot/.style={color=black,only marks,mark=*}} \pgfplotsset{holdot/.style={color=black,fill=white,only marks,mark=*}}

%Here I define some theorem styles and shortcut commands for symbols I use often
\theoremstyle{definition}
\newtheorem{defn}{Definition}
\newtheorem{thm}{Theorem}
\newtheorem{cor}{Corollary}
\newtheorem*{rmk}{Remark}
\newtheorem{lem}{Lemma}
\newtheorem*{joke}{Joke}
\newtheorem{ex}{Example}
\newtheorem*{soln}{Solution}
\newtheorem{prop}{Proposition}

\newcommand{\lra}{\longrightarrow}
\newcommand{\ra}{\rightarrow}
\newcommand{\surj}{\twoheadrightarrow}
\newcommand{\graph}{\mathrm{graph}}
\newcommand{\bb}[1]{\mathbb{#1}}
\newcommand{\Z}{\bb{Z}}
\newcommand{\Q}{\bb{Q}}
\newcommand{\R}{\bb{R}}
\newcommand{\C}{\bb{C}}
\newcommand{\N}{\bb{N}}
\newcommand{\M}{\mathbf{M}}
\newcommand{\m}{\mathbf{m}}
\newcommand{\MM}{\mathscr{M}}
\newcommand{\HH}{\mathscr{H}}
\newcommand{\Om}{\Omega}
\newcommand{\Ho}{\in\HH(\Om)}
\newcommand{\bd}{\partial}
\newcommand{\del}{\partial}
\newcommand{\bardel}{\overline\partial}
\newcommand{\textdf}[1]{\textbf{\textsf{#1}}\index{#1}}
\newcommand{\img}{\mathrm{img}}
\newcommand{\ip}[2]{\left\langle{#1},{#2}\right\rangle}
\newcommand{\inter}[1]{\mathrm{int}{#1}}
\newcommand{\exter}[1]{\mathrm{ext}{#1}}
\newcommand{\cl}[1]{\mathrm{cl}{#1}}
\newcommand{\ds}{\displaystyle}
\newcommand{\vol}{\mathrm{vol}}
\newcommand{\cnt}{\mathrm{ct}}
\newcommand{\osc}{\mathrm{osc}}
\newcommand{\LL}{\mathbf{L}}
\newcommand{\UU}{\mathbf{U}}
\newcommand{\support}{\mathrm{support}}
\newcommand{\AND}{\;\wedge\;}
\newcommand{\OR}{\;\vee\;}
\newcommand{\Oset}{\varnothing}
\newcommand{\st}{\ni}
\newcommand{\wh}{\widehat}
%Pagination stuff.
\setlength{\topmargin}{-0.75in}
\setlength{\oddsidemargin}{0in}
\setlength{\evensidemargin}{0in}
\setlength{\textheight}{9.in}
\setlength{\textwidth}{6.5in}
\pagestyle{empty}
\begin{document}
\begin{flushleft}
Name:\underline{\hspace{13cm}}Date:\underline{\hspace{2cm}}
\end{flushleft}
\begin{center}
{\Large Math 1041-012 \hspace{0.5cm} Section 3.7}
\end{center}
%\vspace{0.2 cm}

\begin{tcolorbox}
\subsection*{Vocabulary}
\begin{itemize}
    \item \textbf{Average Rate of change of $y$ with respect to $x$}:
    \[
    \frac{\Delta y}{\Delta x}=\hspace{4in}
    \]
    \item \textbf{Instantaneous Rate of change of $y$ with respect to $x$}:
    \[
    \frac{dy}{dx}=\hspace{4in}
    \]
\end{itemize}
\end{tcolorbox}
\subsection*{Example 1} The position of a particle is given by
\[
s(t)=t^3-6t^2+9t-10\textrm{ meters.}
\]
Answer the following!
\begin{enumerate}[label=(\alph*)]
    \item Find the average velocity of the particle over the time interval $[0,1]$ seconds? \vspace{2cm}
    \item Find the velocity at time $t$.\vspace{1cm}
    \item What is the velocity after 2 seconds? 4 seconds? What does the sign mean?\vspace{2cm}
    \item When is the particle at rest?\vspace{2cm}
    \clearpage
    \item When is the particle moving forward? Backward?\vspace{4cm}
    \item Find the total distance traveled by the particle during the first five seconds.\vspace{3cm}
    \item Find the acceleration after 4 seconds!\vspace{1cm}
\end{enumerate}
\subsection*{Example 2} The graph of the velocity $v(t)$ of a particle (in meters per second) is shown below!
\begin{center}
\begin{tikzpicture}[scale=1]
\begin{axis}[xlabel={$t$},ylabel={$v$},
  axis x line=middle, axis y line=middle,
  ymin=-2, ymax=3.5, ytick={-2,...,4}, 
  xmin=0, xmax=11.5, xtick={0,...,11},
  axis line style={->}, width=12cm,height=8cm
]
\addplot[domain=0:1,black]{x^2};
\addplot[domain=1:3,black]{-(x-2)^2+2};
\addplot[domain=3:5,black]{(x-4)^2};
\addplot[domain=5:7,black]{-(x-4)+2};
\addplot[domain=7:9,black]{(x-7)^2-1};
\addplot[domain=9:10,black]{-(x-9)^2+3};
\addplot[domain=10:11,black]{-2*x+22};
%\addplot[->,domain=6:10,black]{1/(x-6)};
\end{axis}
\end{tikzpicture}
\end{center}
\begin{enumerate}[label=(\alph*)]
\begin{minipage}{0.6\linewidth}
    \item When is the particle at rest?\vspace{0.75cm}
    \item When is the particle moving forward?\vspace{0.75cm}
    \item When is the particle moving backward?\vspace{0.75cm}
    \item What is the maximum speed? minimum speed?
\end{minipage}
\begin{minipage}{0.5\linewidth}
    \item When is the particle changing direction?\vspace{0.75cm}
    \item When is the particle slowing down?\vspace{0.75cm}
    \item When is the particle speeding up?\vspace{0.75cm}
\end{minipage}
\end{enumerate}
\clearpage
\subsection*{Example 3} The height in meters of a projectile shot vertically upward from a point 6 meters above ground level with an initial velocity 30 meter per second is given by 
\[
h(t)=6+30t-4.9t^2\textrm{ after $t$ seconds.}
\]
\begin{enumerate}[label=(\alph*)]
    \item Find the velocity after 2 seconds, and 4 seconds.\vspace{3cm}
    \item When does the object reach maximum height?\vspace{3cm}
    \item What is the maximum height of the object?\vspace{3cm}
    \item When does it hit the ground? with what velocity does it hit the ground?\vspace{3cm}
\end{enumerate}
\clearpage
\subsection*{Example 4} A spherical balloon is being inflated. The surface area of a sphere is given by $S=4\pi r^2$, where the radius is in feet.
\begin{enumerate}[label=(\alph*)]
    \item Write a formula for the \textit{average} rate of change of the surface area with respect to the radius from radii $r_1$ to $r_2$.\vspace{2.5cm}
    \item Find the average rate of change of the surface area when the radius changes from one foot to two feet.\vspace{2.5cm}
    \item Find the instantaneous rate of change of the surface area with respect to the radius, when the radius is three feet.\vspace{2cm}
\end{enumerate}
\subsection*{\underline{U Try}!} A spherical balloon is being inflated. The volume of a sphere is given by $V=\frac{4}{3}\pi r^3$, where the radius is in feet.
\begin{enumerate}[label=(\alph*)]
    \item Write a formula for the \textit{average} rate of change of the volume with respect to the radius from radii $r_1$ to $r_2$.\vspace{2.5cm}
    \item Find the average rate of change of the volume when the radius changes from one foot to two feet.\vspace{2cm}
    \item Find the instantaneous rate of change of the volume with respect to the radius, when the radius is three feet.
\end{enumerate}
\end{document}